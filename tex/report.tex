\documentclass[12pt]{article}

\usepackage{preamble}

\begin{document}

\textbf{EN.580.475.}

Daniel Yao

\today

\textbf{Introduction.}

Brawl Stars is a 3v3 mobile game by Supercell. The competitive format proceeds in two stages: the draft and the gameplay.

First, the game mode (there are six total) is revealed. Two teams, Blue and Red, each independently choose 3 of 97 brawlers to ban. These bans are then revealed. Blue chooses one brawler, Red chooses two brawlers, Blue chooses two brawlers, and then Red chooses one brawler. These brawlers are chosen without replacement so that each team has three brawlers and that these brawlers are unique. The players then play the game mode with their chosen brawlers, best of three.

In the past few years, the competitive community has come to jokingly refer to the game as "Draft Stars", due to the increasing importance of the draft stage over the gameplay stage with the addition of new characters.

I address this draft stage. First, I train an neural network to predict the victor based on a draft. I then propose a mini-max algorithm to determine the optimal draft order. Finally, I address the (many) limitations of my method.

\textbf{Data.}

I obtained the data during summer 2024 through BrawlStarsAPI. There are 5514 games with 82 unique brawlers. (There were fewer brawlers in the game at the time.) Fear not, dear TA! Though the data are old, the neural network analysis is new.

\textbf{Attempt 1.}

My first attempt is with One-Hot category encoding for the modes and brawlers. There are 6 modes and 82 choices for six players, for a total of 
$$6 + 6(82) = 498$$
dimensions. I train an neural network with two hidden layers of 256 neurons, ReLU activations, sigmoid output, and BCE loss.

\begin{center} \includegraphics{"../output/loss_curve.png"} \end{center}

We see that this neural network can predict the victor with better than average chance, but it grossly overfits the training data. Indeed, the training loss converges after 500 epochs, but the testing accuracy converges after only 100 epochs.

\begin{center} \includegraphics{"../output/confusion_matrix.png"} \end{center}

Nevertheless, it is impressive that 

\end{document}